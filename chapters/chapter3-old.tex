% \begin{savequote}[75mm]
% This is some random quote to start off the chapter.
% \qauthor{Firstname lastname}
% \end{savequote}

\chapter{More than a Routing Tool}

Advanced driver-assistance systems (ADAS) have become ubiquitous in modern vehicles because of the recent developments in communication and sensor technologies. They are primarily developed to improve driving performance, and car and road safety by providing automation and adaptive capabilities to vehicle systems. One of the most widely used tool for driver assistance are automotive navigation systems, which were initially designed to provide digital maps, route guidance for the shortest path to a destination, and traffic incident information \cite{Mikami1978CACS-UrbanControl}. As more private vehicles occupied our roads and more cities are being designed to accommodate and regulate their widespread use, modern automotive navigation systems now also provide information on the cheapest and fastest routes, and traffic condition.

Today, more than half of the world's population call cities their home due to urbanization and a rising middle class \cite{UnitedNations2017}. As we see a consequential increase in car ownership, our efforts in promoting and ensuring sustainable cities are at stake. With dense urban districts and complex road infrastructures, persistent traffic congestion poses a negative effect on our productivity, health, environment, and social equity \cite{Mehndiratta2017}. The worsening traffic conditions have compelled drivers to circumnavigate congested roads and several solutions have been introduced to address this growing problem. Intuitively, cities invest heavily on improving and increasing road network capacity; but adding more links between origin-destination pairs was proven to be counterintuitive and may cause longer travel times \cite{Braess2005,Afimeimounga2005}. 

Another approach was to efficiently manage traffic flow on existing road infrastructures by connecting current fleets to Intelligent Transportation Systems (ITS). Cities have already invested heavily on ITS infrastructure such as toll gantries, adaptive traffic signals, variable-message signs, and traffic detection systems, among others -- all aimed to regulate road use, to capture and provide situational information to drivers, and to redirect them from congested routes. At the same time, in-car navigation and other advanced driver-assistance systems are continually becoming more context-aware -- communicating with other vehicles, the ITS infrastructure, and other smart devices, as well as detecting its immediate environment \cite{Alghamdi2012, Monreal2014, Ali2018}. However in some cases,  in-car navigation systems are barely used and noticed \cite{J.D.Power2012VehicleDeclines}, are becoming too complex to operate \cite{J.D.Power2017ImprovementsFinds}, are not always updated with the latest maps, and sometimes without access to real-time traffic information, which directly impacts their adoption and forcing drivers to find other options.

In the absence and or shortcomings of in-car navigation systems on some vehicle models, smartphone navigation applications such as Waze and Google Maps, have become a preferred alternative for drivers who experience traffic congestion on a daily basis. In the App Annie Rankings \cite{2018GoogleAnnieb}, Google Maps has consistently been the top choice since its introduction of GPS turn-by-turn navigation in 2008. Meanwhile, Waze reported in 2016 that they are already being used in 185 countries by more than 65 million monthly active users \cite{Waze2016DriverIndex}. Other popular navigation applications include HERE WeGo, MapQuest and Bing Maps, and in other countries like Japan, Navitime has been a long time favorite. These navigation applications are free to use and has the latest maps. With the improved sensors in smartphones, these navigation applications started using floating car data from online users to estimate traffic conditions and uses that to suggest optimal driving routes. Maximizing connected drivers, Waze crowd-sources traffic and accident reports, and advisories of police presence, speed traps, and road closures to supplement its turn-by-turn navigation \cite{Levine2014SystemExchange, Valdes-Dapena2016MostDirections}, setting it apart from traditional navigation systems while supporting the notion of navigation as a social activity among drivers and navigators \cite{Forlizzi2010WhereTurn}. At its core, modern in-car navigation systems and navigation applications are routing services, but they are also considered recommender systems because of their sophisticated recommendation engines that use actual and or average road speeds for calculating fastest routes, and learn new routes to suggest to other drivers \cite{2018RoutingServer}. These information on existing road infrastructure and driving behavior have inspired governments to consider their use in influencing future mobility patterns \cite{Ben-Elia2015ResponseReview,Attard2016TheSystems}.Route information and navigation guidance provided by modern navigation applications can be redesigned to motivate drivers to choose unselfish routes.

In this chapter, I make a systematic review of popular driving navigation applications, namely Google Maps, Waze, Apple Maps, Here Maps and OSMAnd. Specifically, I want to:

\begin{enumerate}
\item make an inventory and compare their driving navigation features;
\item compare the kind of route recommendations that they give and how they are presented to the users;
\item compare their voice guidance features and assess how they support self-efficacy in driving; and 
\item reflect on potential gaps in current designs in order to drive future innovations.
\end{enumerate}

This study is still ongoing.