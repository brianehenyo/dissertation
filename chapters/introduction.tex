\chapter{Introduction}
\label{introduction}
In 2050, we will see almost 70\% of the global population move to cities, increasing car ownership and potentially affecting our goals of achieving sustainability. These additional vehicles will slowly congest denser urban environments and complex road networks, worsening traffic conditions and bring forth a number of negative consequences \cite{Mehndiratta2017}. While our current road networks and transportation systems are still keeping up with the rising demand, modern navigation applications such as Waze and Google Maps, and in-car navigation systems found in modern car models today are offering a slight reprieve in dealing with daily traffic conditions. 

As their core routing service becomes more advanced with machine learning and sensing capabilities, navigation applications are becoming integral in many commutes to monitor regular routes, to discover new ones and sometimes to avoid traffic congestion. These modern tools are free to download on most smartphones, have the latest maps, and with some utilizing the Intelligent Transportation Systems of advanced cities. Maximizing built-in sensors and modern GPS, they collect floating car data to learn traffic conditions and further augments these with crowd-sourced reports from ordinary users \cite{Levine2014SystemExchange, Valdes-Dapena2016MostDirections}. All these information are fed into machine learning algorithms to produce models that can power their sophisticated routing algorithms, allowing drivers to cut through traffic by sometimes suggesting unfamiliar routes and small, residential roads. 

\section{Negative Externalities}
Since the first GPS devices were made commercially available to consumers, engineers and designers have always centered their features around the individual driver. Understandably, this has resulted to their modern commercial success as most modern car models include in-car navigation systems by default. For those who do not have that luxury, they can still download free navigation applications like Google Maps and Waze, which also offer turn-by-turn voice guidance. 

However, the widespread use of such applications seem to have negative externalities as shown in the recent work of Bayen et. al. \cite{Thai2016NegativeApproach}. In an agent-based model simulation, they have shown that as more drivers follow the shortcuts provided by navigation applications, smaller residential roads that run parallel or connected to highways experience unlikely congestion. Unlike traffic congestion on highways which can dissipate fast, these small roads will experience congestion for longer because of their low carrying capacity. Insights from this model are also supported by anecdotal evidence of cities like Los Angeles experiencing local unprecedented disruptions because of drivers using Waze \cite{Battelle2016TheShift, Thornton2015HowTimes, Wirtschafter2017DrivingKALW}. These unexpected negative effects have prompted some local communities to start gaming the system \cite{Weise2017WazeAlgorithms} and some government officials to take legal action \cite{Farivar2018LATechnica}. And while these seemingly beneficial applications are allowing individual drivers to avoid traffic congestion or get to their destinations faster, it is worth looking into how we can include other stakeholders into the design of such applications and their algorithms to safeguard the interest of the communities where they operate.

\section{Imagining a Distributed Future}
Because of their fast technological advancements and ubiquity, many government stakeholders are optimistic of the potential of navigation applications in shaping sustainable driving behaviors \cite{Attard2016TheSystems}. If we could rethink current features and designs to go beyond a user-centered approach and include a more systems-centered evaluation of their impacts, we believe that navigation applications has the potential for more. We envision a future in which governments can manage traffic flow on their roads by recommending unselfish routes to drivers. At the same time, we envision future navigation applications to promote these unselfish routes but still give drivers a sense of agency in their navigation decisions. This could have greater impacts in free applications where there is higher chance of mass adoption. Although there are many open challenges on data sparsity and in ensuring the integrity of crowd-sourced reports \cite{Silva2016UsersOpportunities,QingYang2015TowardNetworks,Vyroubal2016MobileSystems} to fully realize this vision, we focus on improving the human-computer interaction and explore a new way of evaluating their system-wide impact.

\section{Interaction with Navigation Applications Today}
In recent years, in-car navigation systems and their mobile application counterparts have gained popularity among drivers \cite{2018GoogleAnnieb, Waze2016DriverIndex}, especially those driving in cities and other urban areas with increasingly complex road networks. Looking at the user's experience, several studies have found older drivers experiencing difficulties following the voice navigation \cite{Dingus1997a,Mahmud2009UserDrivers} while younger drivers overly rely on the turn-by-turn navigation \cite{Mahmud2009UserDrivers}. More recently, Brown \& Laurier enumerated five \emph{normal, natural troubles} of driving with GPS devices with regards to defining destinations, quality of routes, accuracy of maps and sensors, timing and relevance of instructions, and legality of recommendations \cite{Brown2012TheGPS}. All these have profound effects on achieving a positive experience.

By default, drivers are recommended the fastest route to their destinations, with alternative routes either shown up front (i.e. Google Maps) or hidden for you to discover (i.e. Waze). While many people agree and say that they do want fast or short routes when asked at any given day, asking them again in actual driving contexts shows otherwise \cite{Pfleging2014ExperienceNavigation}. This is further supported by empirical evidence from GPS tracks and recorded actual trips that show drivers' repeated non-preference of recommended fastest routes \cite{Quercia2014, Zhu2015DoPrinciple,Tang2016AnalyzingData} and sudden deviations\cite{Fujino2018DetectingTracks,Brown2012TheGPS,Samson:2019:EFI:3290605.3300601}. While there is great support for drivers to make decisions before starting a trip, there are gaps in current systems and applications that fail to consider their changing needs, contexts and preferences, which ultimately affect their compliance on the recommendation.

Looking at these challenges and how we want future navigation applications to become tools in encouraging sustainable driving behaviors, we see two main challenges needed to be addressed:
\begin{itemize}
    \item How do we encourage drivers to choose unselfish routes before a trip?
    \item How do we make sure that drivers continue to follow an unselfish route (if they choose to do so) or convince them to switch to an unselfish route in the middle of a trip?
\end{itemize}

\section{Supporting Instructed Actions}
In Brown \& Laurier's work\cite{Brown2012TheGPS}, after they describe the common dilemmas faced by drivers and passengers with traditional GPS devices, they argued that in order for drivers to have more positive experiences and better \textit{instructed actions}, developers should focus more on supporting their interpretation and analysis of new route guidance and information. Instead of assuming that drivers have zero knowledge, navigation applications should allow them figure out what to do next.

\section{Contribution}
In this dissertation, I focus on the critical HCI question of how to encourage drivers to follow unselfish routes for their daily commutes. I explore techniques of giving navigation information and guidance for drivers with a focus on two approaches: 1) exploring novel and persuasive interaction techniques and information design that supports making informed decisions during navigation towards unselfish, purposeful, and effective driving behaviors, and 2) creating agent-based models to investigate on a broader scale how these techniques might affect traffic flow when used in daily commutes. 

I begin with evaluating the features of existing commercial navigation systems and applications based on how they support self-efficacy, sense of agency and behavior change. I follow this with an observational study of drivers using navigation applications in their daily commutes. We found that while drivers choose a recommended route in urgent situations, many still preferred to follow familiar routes. They also made deviations from their original choice because of unfamiliar roads, lack of local context, perceived driving unsuitability, and inconsistencies with realized navigation experiences. From these initial works, I conducted two studies that evaluated prototypes that aim to help drivers make better informed navigation decisions. The first study explores the effects of adding explanations and other cues to route recommendations and route information when a driver chooses a route to follow before driving. 

During a trip, traffic conditions along a chosen route might change. The second study explores the use of two-party conversations between voice agents in giving alternative turns or routes. We found that while most drivers followed directions appropriate for the given scenarios, they were more likely to make inappropriate choices after hearing alternatives in conversations. On the other hand, two-party conversations allowed drivers to better reflect on their choices after trips.

\section{Structure}
We begin this dissertation by reviewing the related works that investigate how drivers interact with navigation systems and the factors affecting their route choice (Chapter 2). We then review the top navigation applications to identify common core design features and assess the route recommendations that they give and the way they are presented (Chapter 3). In Chapter 4, we discuss how drivers interact with modern navigation applications. Then in Chapter 5, we explore the use of the Self-Determination Theory in designing autonomy-supportive systems and investigate the effects of adding motivative and familiarity information to encourage the choice unselfish routes. In Chapter 6, we focus on the on-trip voice guidance and explore the use of two-way conversations to influence route choice. Then in Chapter 7, we use the findings and behaviors learned from previous chapters to assess how these techniques can affect the state of a simple road network from a mesoscopic and macroscopic level. We conclude this dissertation with a summary of my key contributions as well as an envisioning of future navigation systems and applications. (Chapter 8).