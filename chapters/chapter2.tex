% \begin{savequote}[75mm]
% Nulla facilisi. In vel sem. Morbi id urna in diam dignissim feugiat. Proin molestie tortor eu velit. Aliquam erat volutpat. Nullam ultrices, diam tempus vulputate egestas, eros pede varius leo.
% \qauthor{Quoteauthor Lastname}
% \end{savequote}

\chapter{Related Works}

\section{Interacting with Recommender and Navigation Systems}
With the incredible amount of data from digital and social media, and those from connected devices and sensors in the Internet of Things, recommender systems have been a boon to digital natives in making sense of and discovering new information. This popularity has gained significant attention to its evaluation in HCI, especially for a more user-centric approach. Knijnenberg et. al. \cite{Knijnenburg2012ExplainingSystems} evaluated collaborative filtering recommender systems and found that increased usage is strongly correlated to a positive personalized experience, but their perceptions, experiences and behaviors change over time. These are also influenced by personal and situational characteristics such as age, gender and domain knowledge. Additionally, they found that when users perceive a recommendation set as more diverse, they see it as more accurate and less difficult to choose from. This is echoed by Ekstrand et. al. \cite{Ekstrand2014UserAlgorithms} when they found users choose a system with more diverse recommendations. They also emphasized the importance of building trust in the early use of recommender systems as their results show negative effects of novelty. Comparing between collaborative, content-based and hybrid recommender systems, Wu \cite{Wu2015HybridSystems} found that users mostly preferred recommendation sets that use hybrid filtering. In particular, users see more benefit in recommendations that match their own behavior history (content-based) than those that match the history of similar users (collaborative). Moving to a different type of system, Rong and Pu \cite{Hu2010ASystems} developed a personality-based recommender system and found that novice users had an easier time building their profiles using personality quizzes because it doesn't need much domain knowledge. When users were asked to build profiles for themselves and their friends, they perceived the recommendation for their friends as more accurate. Much of these works have focused on user perceptions and behaviors towards the main approaches to recommender systems with a single criterion for matching, and they have demonstrated user-centric evaluations besides algorithmic accuracy. However, further analysis is needed for the growing number of mobile and ubiquitous recommender systems that incorporate multi-criteria preferences, probabilistic models, and temporal, spatial and crowd-sourced information.

With a focus on GPS devices, Dingus et. al. \cite{Dingus1997a} did camera and instrumented car studies for drivers who use TravTek. They found that older drivers have a difficult time driving and navigating, and despite being more careful, they still made more safety-related errors. Generally, drivers benefited most when using turn-by-turn guidance with voice, resulting to less glances to the device and faster travel times. In their naturalistic field study, most drivers used the GPS device in their rental cars. Al Mahmud et. al. \cite{Mahmud2009UserDrivers} also found old drivers having difficulties with in-car GPS. As a result, they tend to not follow it completely due to reliability concerns and high amount of instructions. On the other hand, the younger drivers were found to be too dependent at times. Lastly, Brown \& Laurier's study \cite{Brown2012TheGPS} documented five problems that drivers usually encounter with their GPS devices and considers it a skilled activity. In order for a driver to have a positive experience and make suitable \emph{instructed actions}, other than giving focus on providing very detailed instructions which can overwhelm and cause more confusion, it is equally important to support the driver's interpretation and analysis of an instruction or new information as they move and figure out what to do next. Clearly, these works have shown how driving and navigating performance is affected by the use of early smartphone, dashboard-mounted and in-car GPS devices. But with a new generation of navigation applications that dynamically adjusts to real-time and historical contextual information, and provides sets of crowd-sourced information, further analysis is needed to see whether there are changes in navigating practice and decision making, and whether they are associated with the type of trip, trip context, and road conditions.

\section{Potential for Behavioral Adaptation}
Because of the ubiquity, cost-effectiveness, and positive utility of smartphone navigation applications, there is growing optimism of their potential in improving urban participatory sensing \cite{Silva2013TrafficAlerts,Xie2015AnNetworks,Silva2016UsersOpportunities} and in shaping sustainable mobility patterns among driving citizens \cite{Ben-Elia2015ResponseReview,Attard2016TheSystems}. There are three categories of travel information that can affect travel behavior, namely experiential, descriptive, and prescriptive \cite{Ben-Elia2015ResponseReview}. Experiential information are provided as feedback or repeated information from previous experiences, while descriptive information depict current conditions based on historic or real-time data such as estimated times of arrival and traffic conditions. Utilizing experiential and descriptive information, prescriptive information can come as suggestions (e.g. shortest, fastest, and cheapest routes) and or guidance (e.g. turn-by-turn directions). Nowadays, most modern navigation applications provide descriptive and prescriptive information as their main features \cite{Sha2013SocialNavigation}. In Chorus's \cite{Chorus2006TravelReview} and Ben-Elia's \cite{Ben-Elia2015ResponseReview} literature reviews, they have highlighted the extensive focus of recent works on the positive effects of experiential and descriptive information to influence the travel behavior of car drivers. Experiential information has been proven helpful in adapting to uncertain conditions, while descriptive information is particularly valuable in coping with non-correlated and Black Swan events like road accidents and sudden bad weather. However, there is still relatively few studies about the implications of prescriptive information.

\section{Route Choice and Driver's Compliance}
Developers have so far focused on the assumption that drivers would always follow the fastest route to a destination. For most navigation applications, drivers are provided with a number of recommended routes based on a criteria and they can select which one to follow. By default, the fastest route criteria is set unless customizations are made. In the case of Waze, it immediately starts the turn-by-turn navigation and leaves it to the user to check alternative options \cite{Levine2014SystemExchange}. However, this doesn't seem to be the case based on studies examining GPS track data. Zhu and Levinson \cite{Zhu2015DoPrinciple} noticed from GPS tracks that drivers do not always choose the shortest path in their daily commutes. In the follow up work of Tang et. al. \cite{Tang2016AnalyzingData}, some drivers even take a different route each day for their commutes. Recognizing that desired driving experiences have an influence on route choice and vice versa, Pfleging et. al.'s \cite{Pfleging2014ExperienceNavigation} web survey show that the most considered factor for drivers is whether it is the fastest route, but when asked to choose a route from work to home using a prototype navigation screen, 49.1\% chose the fuel-efficient route. Only 18.4\% and 3.5\% chose the fastest and shortest routes, respectively. While these provide rich empirical evidence, it is not clear whether the same prioritization and decision making holds true in real driving scenarios under different circumstances.

Relatedly, Fujino et. al. \cite{Fujino2018DetectingTracks} conducted a more recent study to investigate the phenomena of drivers deviating from the recommended optimal routes of in-car navigation systems and where they usually happen. They analyzed GPS tracks that were collected over 4 years within a 20km\textsuperscript{2} area in Kyoto, Japan. They found that drivers have made significant deviations on intersections with poor on-road signages and those near tourist areas. They also speculated on possible reasons for the deviations based on the physical characteristics of the intersections. While these studies already provide empirical evidence on the surprising route choice and non-compliant behaviors of drivers, none of them had prior knowledge whether the observed drivers used prescriptive information from in-car navigation systems or navigation applications. In the case of \cite{Zhu2015DoPrinciple, Tang2016AnalyzingData, Fujino2018DetectingTracks}, they had no information on the intended route of the drivers nor do they know if the drivers were initially following the guidance of the in-car navigation system used to collect the GPS tracks. Thus, further investigation is warranted to understand why drivers deviated from the recommended optimal routes and whether they chose a recommended route in the first place.

In HCI, Brown \& Laurier's study \cite{Brown2012TheGPS} also noted instances of drivers not following GPS recommendations from their corpus of naturalistic video data. They argue that GPS use is rather a skilled activity as drivers need competency to overcome the \emph{normal, natural troubles} that GPS devices make. Several of these problems such as complex routes, superfluous instructions, map and sensor inaccuracies, and timing of instructions, offer a glimpse as to why GPS recommendations are not followed. Addressing the complex route problem, Patel et. al. \cite{Patel2006PersonalizingRoutes} found that drivers prefer simplified route instructions using familiar landmarks.

As more drivers use descriptive and prescriptive information from navigation applications and more government stakeholders seek to use them in managing road networks, it is crucial that navigation applications become successful in shaping the travel behavior of connected drivers. Ali et. al. \cite{Ali2018} argues that behavioral adaptation is directly affected by the degree of compliance a driver has with the information provided by navigation applications. Although they are referring to connected vehicle technologies, the same assertion can also be made for navigation applications because they provide the same kind of information. It is worth exploring how we can better utilize descriptive information and present prescriptive information to create navigation experiences that encourages behavioral adaptation.

% For an example of a full page figure, see Fig.~\ref{fig:myFullPageFigure}.

%% Requires fltpage2 package
%%
% \begin{FPfigure}
% \includegraphics[width=\textwidth]{figures/fullpage}
% \caption[Short figure name.]{This is a full page figure using the FPfigure command. It takes up the whole page and the caption appears on the preceding page. Its useful for large figures. Harvard's rules about full page figures are tricky, but you don't have to worry about it because we took care of it for you. For example, the full figure is supposed to have a title in the same style as the caption but without the actual caption. The caption is supposed to appear alone on the preceding page with no other text. You do't have to worry about any of that. We have modified the fltpage package to make it work. This is a lengthy caption and it clearly would not fit on the same page as the figure. Note that you should only use the FPfigure command in instances where the figure really is too large. If the figure is small enough to fit by the caption than it does not produce the desired effect. Good luck with your thesis. I have to keep writing this to make the caption really long. LaTex is a lot of fun. You will enjoy working with it. Good luck on your post doctoral life! I am looking forward to mine. \label{fig:myFullPageFigure}}
% \end{FPfigure}
% \afterpage{\clearpage}