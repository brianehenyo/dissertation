% \begin{savequote}[75mm]
% Nulla facilisi. In vel sem. Morbi id urna in diam dignissim feugiat. Proin molestie tortor eu velit. Aliquam erat volutpat. Nullam ultrices, diam tempus vulputate egestas, eros pede varius leo.
% \qauthor{Quoteauthor Lastname}
% \end{savequote}

\chapter{Related Works}
\label{ChapterRL}

Advanced driver-assistance systems (ADAS) have become ubiquitous in modern vehicles because of the recent developments in communication and sensor technologies. They are primarily developed to improve driving performance, and car and road safety by providing automation and adaptive capabilities to vehicle systems. One of the most widely used tool for driver assistance are automotive navigation systems, which were initially designed to provide digital maps, route guidance for the shortest path to a destination, and traffic incident information \cite{Mikami1978CACS-UrbanControl}. As more private vehicles occupied our roads and more cities are being designed to accommodate and regulate their widespread use, modern automotive navigation systems now also provide information on the cheapest and fastest routes, and traffic condition.

Today, more than half of the world's population call cities their home due to urbanization and a rising middle class \cite{UnitedNations2017}. As we see a consequential increase in car ownership, our efforts in promoting and ensuring sustainable cities are at stake. With dense urban districts and complex road infrastructures, persistent traffic congestion poses a negative effect on our productivity, health, environment, and social equity \cite{Mehndiratta2017}. The worsening traffic conditions have compelled drivers to circumnavigate congested roads and several solutions have been introduced to address this growing problem. Intuitively, cities invest heavily on improving and increasing road network capacity; but adding more links between origin-destination pairs was proven to be counterintuitive and may cause longer travel times \cite{Braess2005,Afimeimounga2005}. 

Another approach was to efficiently manage traffic flow on existing road infrastructures by connecting current fleets to Intelligent Transportation Systems (ITS). Cities have already invested heavily on ITS infrastructure such as toll gantries, adaptive traffic signals, variable-message signs, and traffic detection systems, among others -- all aimed to regulate road use, to capture and provide situational information to drivers, and to redirect them from congested routes. At the same time, in-car navigation and other advanced driver-assistance systems are continually becoming more context-aware -- communicating with other vehicles, the ITS infrastructure, and other smart devices, as well as detecting its immediate environment \cite{Alghamdi2012, Monreal2014, Ali2018}. However in some cases,  in-car navigation systems are barely used and noticed \cite{J.D.Power2012VehicleDeclines}, are becoming too complex to operate \cite{J.D.Power2017ImprovementsFinds}, are not always updated with the latest maps, and sometimes without access to real-time traffic information, which directly impacts their adoption and forcing drivers to find other options.

In the absence and or shortcomings of in-car navigation systems on some vehicle models, smartphone navigation applications such as Waze and Google Maps, have become a preferred alternative for drivers who experience traffic congestion on a daily basis. In the App Annie Rankings \cite{2018GoogleAnnieb}, Google Maps has consistently been the top choice since its introduction of GPS turn-by-turn navigation in 2008. Meanwhile, Waze reported in 2016 that they are already being used in 185 countries by more than 65 million monthly active users \cite{Waze2016DriverIndex}. Other popular navigation applications include HERE WeGo, MapQuest and Bing Maps, and in other countries like Japan, Navitime has been a long time favorite. These navigation applications are free to use and has the latest maps. With the improved sensors in smartphones, these navigation applications started using floating car data from online users to estimate traffic conditions and uses that to suggest optimal driving routes. Maximizing connected drivers, Waze crowd-sources traffic and accident reports, and advisories of police presence, speed traps, and road closures to supplement its turn-by-turn navigation \cite{Levine2014SystemExchange, Valdes-Dapena2016MostDirections}, setting it apart from traditional navigation systems while supporting the notion of navigation as a social activity among drivers and navigators \cite{Forlizzi2010WhereTurn}. At its core, modern in-car navigation systems and navigation applications are routing services, but they are also considered recommender systems because of their sophisticated recommendation engines that use actual and or average road speeds for calculating fastest routes, and learn new routes to suggest to other drivers \cite{2018RoutingServer}. These information on existing road infrastructure and driving behavior have inspired governments to consider their use in influencing future mobility patterns \cite{Ben-Elia2015ResponseReview,Attard2016TheSystems}.

\section{Interacting with Recommender Systems}

With the incredible amount of data from digital and social media, and those from connected devices and sensors in the Internet of Things, recommender systems have been a boon to digital natives in making sense of and discovering new information. This popularity has gained significant attention to its evaluation in HCI, especially for a more user-centric approach. Knijnenberg et. al. \cite{Knijnenburg2012ExplainingSystems} evaluated collaborative filtering recommender systems and found that increased usage is strongly correlated to a positive personalized experience, but their perceptions, experiences and behaviors change over time. These are also influenced by personal and situational characteristics such as age, gender and domain knowledge. Additionally, they found that when users perceive a recommendation set as more diverse, they see it as more accurate and less difficult to choose from. This is echoed by Ekstrand et. al. \cite{Ekstrand2014UserAlgorithms} when they found users choose a system with more diverse recommendations. They also emphasized the importance of building trust in the early use of recommender systems as their results show negative effects of novelty. 

Comparing between collaborative, content-based and hybrid recommender systems, Wu \cite{Wu2015HybridSystems} found that users mostly preferred recommendation sets that use hybrid filtering. In particular, users see more benefit in recommendations that match their own behavior history (content-based) than those that match the history of similar users (collaborative). Moving to a different type of system, Rong and Pu \cite{Hu2010ASystems} developed a personality-based recommender system and found that novice users had an easier time building their profiles using personality quizzes because it doesn't need much domain knowledge. When users were asked to build profiles for themselves and their friends, they perceived the recommendation for their friends as more accurate. Much of these works have focused on user perceptions and behaviors towards the main approaches to recommender systems with a single criterion for matching, and they have demonstrated user-centric evaluations besides algorithmic accuracy. However, further analysis is needed for the growing number of mobile and ubiquitous recommender systems that incorporate multi-criteria preferences, probabilistic models, and temporal, spatial and crowd-sourced information.

\section{Ongoing Struggles with Navigation Systems}

With a focus on GPS devices, Dingus et. al. \cite{Dingus1997a} did camera and instrumented car studies for drivers who use TravTek. They found that older drivers have a difficult time driving and navigating, and despite being more careful, they still made more safety-related errors. Generally, drivers benefited most when using turn-by-turn guidance with voice, resulting to less glances to the device and faster travel times. In their naturalistic field study, most drivers used the GPS device in their rental cars. Al Mahmud et. al. \cite{Mahmud2009UserDrivers} also found old drivers having difficulties with in-car GPS. As a result, they tend to not follow it completely due to reliability concerns and high amount of instructions. On the other hand, the younger drivers were found to be too dependent at times. 

Focusing on more portable GPS devices, Brown \& Laurier's study \cite{Brown2012TheGPS} documented five problems that drivers usually encounter during trips and came to the conclusion that navigation with GPS devices is a skilled activity. In order for a driver to have a positive experience and make suitable \emph{instructed actions}, other than giving focus on providing very detailed instructions which can overwhelm and cause more confusion, it is equally important to support the driver's interpretation and analysis of an instruction or new information as they move and figure out what to do next. Clearly, these works have shown how driving and navigating performance is affected by the use of early smartphone, dashboard-mounted and in-car GPS devices. But with a new generation of navigation applications that dynamically adjusts to real-time and historical contextual information, and provides sets of crowd-sourced information, further analysis is needed to see whether there are changes in navigating practice and decision making, and whether they are associated with the type of trip, trip context, and road conditions. 

More recently, Antrobus et. al. investigated how effective the use of SatNav devices are compared to collaborative passengers in helping drivers learn routes and become more aware of their environments while navigating\cite{Antrobus2017Driver-PassengerSystems}. They found that drivers learned the routes better after they drove with a collaborative passenger because they were using more landmark, road sign and dynamic landmark descriptors in telling the next navigation instruction. In contrast, the SatNav was only giving distance descriptors. Additionally, the collaborative passengers were more helpful because they  confirm what the driver is interpreting as the next navigation maneuver, give confidence boosting words to the driver, and provide proper orientation.

Despite the continuous improvement of such navigation tools, although mostly on the digital maps they use, these recent works suggest that drivers continue to experience problems with the provided information and turn-by-turn navigation guidance. Additionally, they have focused on early smartphone, dashboard-mounted and in-car GPS devices. But with a new generation of navigation applications, I’m curious whether there are changes in navigating practice and decision making, and whether they are associated with the type of trip, trip context, and road conditions.

\section{Route Choice Behavior}
Because of the ubiquity, cost-effectiveness, and positive utility of smartphone navigation applications, there is growing optimism of the potential of navigation applications in improving urban participatory sensing \cite{Silva2013TrafficAlerts,Xie2015AnNetworks,Silva2016UsersOpportunities} and in shaping sustainable mobility patterns among driving citizens \cite{Ben-Elia2015ResponseReview,Attard2016TheSystems}. Key to the realization of this potential is the navigation application's ability to influence the route choice behavior of drivers. 

Route choice is a decision making task that actively occurs in driving navigation\cite{donges1999conceptual}. It is a driving behavior that is based on their active consciousness of their surroundings, knowledge of relevant travel information about possible routes, and recollection of past navigation experiences. According to Ben-Elia and Avineri, there are three categories of travel information that can affect travel behavior, namely experiential, descriptive, and prescriptive\cite{Ben-Elia2015ResponseReview}. Experiential information are provided as feedback or repeated information from previous experiences. In actual implementations, they are passively captured in the background but are mainly used to train machine learning models required to improve digital maps and to provide route recommendations for all users. Thus, experiential information has never been utilized for personalized navigation experiences and drivers still rely on their cognitive functions to retrieve experiential information from memory. On the other hand, descriptive information depicts current conditions based on historic or real-time data such as estimated times of arrival and traffic conditions. Utilizing experiential and descriptive information, prescriptive information can come as suggestions (e.g. shortest, fastest, and cheapest routes) and or guidance (e.g. turn-by-turn directions) to help drivers optimize their travel time and positive driving navigation experience. Nowadays, most modern navigation applications provide descriptive and prescriptive information as their main features to inform and redirect drivers\cite{Sha2013SocialNavigation}. In the absence of in-car navigation systems and navigation applications, drivers can also access these information through variable message signs, which is another type of advanced traveller information system (ATIS) that are physically installed on many major roads in cities. Unselfish routes are typically considered and provided as prescriptive information in many route choice studies and in some modern navigation applications. And even so, there is still relatively few studies about the implications of prescriptive information on route choice. 

In Chorus et. al.'s \cite{Chorus2006TravelReview} and Ben-Elia \& Avineri's \cite{Ben-Elia2015ResponseReview} surveys of literature, they have highlighted the extensive focus on the positive effects of experiential and descriptive information to influence the travel behavior of car drivers. Experiential information has been proven helpful in adapting to uncertain conditions, while descriptive information is particularly valuable in coping with non-correlated and Black Swan events like road accidents and sudden bad weather. As a universal behavior, Abdel-Aty and Abdalla found initial evidence from a small percentage of their participants who showed more instances of deviation from a regular route when they had access to travel information\cite{abdel2004modeling}. In addition, drivers were also shown to be more likely to change and follow a recommended route when they see it provided by variable message signs along the road\cite{peeta2006driver,erke2007effects}. 

But the mere provision of such information are sometimes not enough. Route choice and compliance were also shown to be dependent on the quality of route information\cite{Chorus2006TravelReview}. In an early study, Chen et. al. conducted an analysis on the effects of information quality and credibility on drivers' compliance to travel information provided by advanced traveler information systems\cite{chen1999effect}. In an interactive multi-user simulation environment, three aspects of information strategies were tested on participants. First is the nature of information (descriptive or prescriptive). Second is information quality, which are based on six levels of precision, from very precise travel time estimates to random values. Third and last is post-trip feedback. They found that drivers show high compliance when drivers are provided with prescriptive information with very precise travel time estimates. In practice, pre-trip travel times do not sometimes match the realized travel times because of dynamic traffic situations which were not accounted for at the beginning. But since travel time reliability is a major consideration for route choice\cite{lotan1997effects,carrion2012value}, several strategies have been explored like showing ranges \cite{tanaka2014experimental} and presenting standard deviations from the mean travel time\cite{bifulco2014evaluating}.

In a conceptual framework for route choice behavior, Bogers et. al. showed the importance of habit, riskiness and presentation of past information\cite{bogers2005joint}. Habit is when drivers learn and regularly use routes that leads to their unconscious selection of the habitual choice and bias against other alternatives. This study also showed that drivers improve their navigation performance when they are shown the realized travel times of all their past trips, as well as the forgone outcomes. This elaborate and historical information acts as a memory aid for drivers, however it cannot be ascertained how this can be effectively shown in practice. Lastly, drivers were shown to be naturally averse to risk. Thus, they would most likely choose a route with high certainty. 

When alternative routes, like side roads, have faster travel time, drivers are more likely to choose them. Ringhand \& Vollrath found in 2019 that even just 20-second gains can get around 20\% more drivers to shift to side roads\cite{ringhand2019effect}. And relative increases in travel time of recommended routes can negatively affect their chances of being selected\cite{abdel2004modeling,ardeshiri2015driving,ringhand2019effect}. But with better familiarity, chances can be levelled and can lead to some positive impact on driver's route choice and compliance\cite{adler2001investigating,bogers2005joint,shiftan2011route,ardeshiri2015driving}. 

Other than travel time and familiarity, other types of travel information also showed effects on route choice. In the work of Ramaekers et. al., they showed that a trip's purpose (e.g. work-related) has an effect, along with trip length\cite{ramaekers2013modelling}. In cities with vast networks of roads and intersections, drivers were shown to heavily consider the effects of traffic lights on their travel time. Regardless of whether the recommended route has a longer distance or travel time, drivers would still choose them if it avoids traffic lights\cite{abdel2004modeling,papinski2009exploring,palat2014numerosity,venigalla2017psychology,Ringhand2017InvestigatingTime,ringhand2018make}. As much as possible, drivers avoid roads or routes with many traffic lights \cite{palat2014numerosity} and those with long waiting times even if there are only a few encounters\cite{Ringhand2017InvestigatingTime}. Although it should also be noted that drivers often underestimate their judgement of waiting times\cite{wu2009perception}. Instead of showing just the actual travel time of two route choices, a recent study also showed that routes that positively frame travel time gains were chosen more than the way drivers avoided routes with negatively framed travel time loses\cite{ringhand2019faster}. When it comes to variable message signs, Peeta \& Ramos found drivers were more willing to follow recommended reroutes when they are shown information about road accidents and travel time delays\cite{peeta2006driver}.

Route choice and navigation in general do not only rely in the type and quality of travel information provided. More often than not, drivers make navigational decisions based on a combination of travel information and external events and factors. In the early work of Gärling et. al., they found that time pressure has a combined effect with information on the recommended route\cite{garling1999role}. But when Ringhand \& Vollrath investigated this further, they found effects of time pressure on decision making time but not so much on route choice\cite{Ringhand2017InvestigatingTime}. But even though some alternative routes are shown to be faster, their chances of being selected are also affected by the  complexity of a route's traffic situation. Considering a variety of factors like speed limit, road and lane widths, intersections, traffic from various directions and sources, disruptions, pedestrian foot traffic, and points of interest, the same authors found that less drivers choose an alternative route when it has high complexity (e.g. has oncoming and pedestrian traffic) even if the route can be faster or shorter\cite{ringhand2019effect}. In addition, Thomas \& Tutert found that the physical properties and layout of roads within a route also play an important factor in route choice\cite{thomas2015route}. For example, routes that include circumferential or orbital roads tend to be chosen more by drivers compared to those that are shorter and passes through the city center.  

All things considered, it can be a cause for concern whether drivers would ever choose unselfish routes for their daily commutes, especially since these are already familiar and regular routes. In a distributed future wherein traffic management systems provide recommendations, unselfish routes would be sub-optimal alternatives with longer travel times and or distances\cite{ringhand2018make}, and is aimed to minimize the marginal cost of one's route choice on other drivers\cite{colak2016understanding}. However, it can also be the case that the recommended unselfish route would be something familiar to the user but seldom used by other drivers. Whichever it may be, designers have to look into applying behavior change techniques in order to increase chances of selection and compliance for unselfish routes. For example, drivers can be nudged to choose unselfish routes by showing them as the default route recommendation, which is similar to how Waze automatically starts its navigation guidance for the fastest route\cite{avineri2009nudging}. Aside from the strategy of highlighting or making default the system optimal route to drivers, designers can also show the context and rationale behind the recommendations. This is under the assumption that if drivers would know why they are being recommended a system optimal route, they could somehow align their decisions with it. One way of doing it is by informing the drivers about the source of the recommendation. In two stated route choice studies where drivers were asked to choose between the fastest route and the system optimal route, they were informed that the system optimal route was given by a traffic management system\cite{kerkman2012car,ringhand2018make}. In Kerman et. al.'s study, they found that route advice was considered more by participants when it showed different attributes of the alternative routes and when it shows that it can support traffic management outcomes. There is more effect when it is labelled personal for the driver. However in their stimuli, there is no indication which among the choices are system optimal. Ringhand \& Vollrath did a two related studies by investigating the combined effects of presenting the source of recommendation and highlighting the system optimal recommendation\cite{ringhand2018make}. When they only highlighted the system optimal recommendation to the participants, their individual compliance of drivers increased by a small fraction. But when they added information about the source of the recommendation and described a hypothetical traffic management system in a followup study, it did not show any effect on route choice unlike in Kerman et. al.   

Recent attempts to nudge drivers into choosing unselfish routes have so far focused on providing information about a hypothetical source of recommendation and on explicitly labeling them as recommended by a traffic management system. Both information strategies seem to appeal to the extent of a driver's altruistic nature. Although results showed that the driver's decision making was partially correlated with their altruism, both strategies were not really designed for behavior change at the onset. In this dissertation, my goal is to explore other types of travel information that are grounded on behavior change theories to achieve desired route choice outcomes. 

\section{Driver's Compliance}
Developers have so far focused on the assumption that drivers would always follow the fastest route to a destination. For most navigation applications, drivers are provided with a number of recommended routes based on a criteria and they can select which one to follow. By default, the fastest route criteria is set unless customization are made. In the case of Waze, it immediately starts the turn-by-turn navigation and leaves it to the user to check alternative options \cite{Levine2014SystemExchange}. However, this doesn't seem to be the case based on studies examining GPS track data. Zhu and Levinson \cite{Zhu2015DoPrinciple} noticed from GPS tracks that drivers do not always choose the shortest path in their daily commutes. In the follow up work of Tang et. al. \cite{Tang2016AnalyzingData}, some drivers even take a different route each day for their commutes. Recognizing that desired driving experiences have an influence on route choice and vice versa, Pfleging et. al.'s \cite{Pfleging2014ExperienceNavigation} web survey show that the most considered factor for drivers is whether it is the fastest route, but when asked to choose a route from work to home using a prototype navigation screen, 49.1\% chose the fuel-efficient route. Only 18.4\% and 3.5\% chose the fastest and shortest routes, respectively. While these provide rich empirical evidence, it is not clear whether the same prioritization and decision making holds true in real driving scenarios under different circumstances.

Relatedly, Fujino et. al. \cite{Fujino2018DetectingTracks} conducted a more recent study to investigate the phenomena of drivers deviating from the recommended optimal routes of in-car navigation systems and where they usually happen. They analyzed GPS tracks that were collected over 4 years within a 20km\textsuperscript{2} area in Kyoto, Japan. They found that drivers have made significant deviations on intersections with poor on-road signages and those near tourist areas. They also speculated on possible reasons for the deviations based on the physical characteristics of the intersections. While these studies already provide empirical evidence on the surprising route choice and non-compliant behaviors of drivers, none of them had prior knowledge whether the observed drivers used prescriptive information from in-car navigation systems or navigation applications. In the case of \cite{Zhu2015DoPrinciple, Tang2016AnalyzingData, Fujino2018DetectingTracks}, they had no information on the intended route of the drivers nor do they know if the drivers were initially following the guidance of the in-car navigation system used to collect the GPS tracks. Thus, further investigation is warranted to understand why drivers deviated from the recommended optimal routes and whether they chose a recommended route in the first place.

In HCI, Brown \& Laurier's study \cite{Brown2012TheGPS} also noted instances of drivers not following GPS recommendations from their corpus of naturalistic video data. They argue that GPS use is rather a skilled activity as drivers need competency to overcome the \emph{normal, natural troubles} that GPS devices make. Several of these problems such as complex routes, superfluous instructions, map and sensor inaccuracies, and timing of instructions, offer a glimpse as to why GPS recommendations are not followed. Addressing the complex route problem, Patel et. al. \cite{Patel2006PersonalizingRoutes} found that drivers prefer simplified route instructions using familiar landmarks.

As more drivers use descriptive and prescriptive information from navigation applications and more government stakeholders seek to use them in managing road networks, it is crucial that navigation applications become successful in shaping the travel behavior of connected drivers. Sharma et. al. \cite{Ali2018} argues that behavioral adaptation is directly affected by the degree of compliance a driver has with the information provided by navigation applications. Although they are referring to connected vehicle technologies, the same assertion can also be made for navigation applications because they provide the same kind of information. It is worth exploring how we can better utilize descriptive information and present prescriptive information to create navigation experiences that encourages behavioral adaptation.

\section{Behavior Theories in HCI}
Human behavior is an action that someone does as a response to antecedents\cite{davis2015theories}. When one responds repeatedly to a situation or stimuli in a similar manner, changing them can be challenging. In order to understand why we stick to regular responses, behavior theories allow us to predict future responses using their underlying concepts, propositions and constructs\cite{glanz2008health}. Several theories have emerged from psychology in order to help explain and predict desired behavioral outcomes in education, health and sports, to name a few.

In this section, I will discuss some behavior theories that have been used extensively in HCI research\cite{hekler2013mind}. First is Social Cognitive Theory which posits that humans can learn new behaviors by observing other people or models performing that desired behavior\cite{bandura2001social}. Typically, learning about the behavior and its consequences happens through social interactions, physical environment, and media exposure. As an ecological theory, it can be used to ground interventions that maximize the behavior change potential of these external influences. Focusing on one's beliefs, the Theory of Planned Behavior posits that one's intention to perform an action or behavior is shaped by their normative belief or attitude, subjective norm, and perceived behavioral control or self-efficacy\cite{ajzen1985intentions}. It links a person's beliefs to their performance of a behavior by incorporating perceived behavioral control with the theory of reasoned action. In a similar manner but specific for the health domain, the Health Belief Model suggests that a person's performance or avoidance of health-promoting behavior can be explained by their perceived susceptibility to and severity of a health problem, perceived benefits of doing the action, perceived barriers to performing the behavior, and self-efficacy\cite{prochaska1983stages,rosenstock1974health}. Besides balancing these beliefs, the health-promoting behavior must also be triggered with a cue to action. Unlike previous theories that focus on external influences and personal beliefs, people following the Goal Setting Theory write an action plan, which is a physical artifact or document that is meant as a memory guide. By referring to the action plan, they can be motivated to perform the intended behavior\cite{locke2002building}. 

Although the aforementioned theories have already been used to implement interventions for behavioral outcomes in different aspects of life (e.g. education, health and well-being, sports, life goals), all of them are focused on personal gains. In this study, my goal is to promote altruistic behavior for drivers by having them choose unselfish routes. Thus, it would remain a challenge if there is only focus on extrinsic aspects of behavior change (e.g. rewards, challenges, influences). If a person does not have an altruistic behavior, we must design strategies that will allow them to have a conscious valuing of the desired behavior until it becomes part of one's self. This process of internalization can improve the quality of motivation, from amotivation to intrinsic motivation.

Motivation is an important construct of most behavior theories and is what moves us into action. However, what most theories do not consider is that motivation varies from person to person\cite{ryan2017basic}. Self-Determination Theory focuses on the motivation aspect of behavior change and introduces different types, sources and orientations that affect the quality of motivation\cite{ryan2000intrinsic}. Specifically, it suggests that people can have intrinsic motivation, amotivation, and a spectrum of extrinsic motivation towards an event or action. The quality and type of motivation can change over time depending on how the environment supports certain basic psychological needs. Because of SDT's focus on internal processes, it has since been used to understand the development of one's motivation and self-determined behavior\cite{wehmeyer2017development}. Lastly, they have been applied for interventions and applications designed for life's various aspects and have shown to deliver long term benefits\cite{friederichs2015long}. Here, I inform my designs using Self-Determination Theory as it allows for better sustainment of the desired behavior (i.e. choosing unselfish routes).

\section{Technologies for Behavior Change}
As technology becomes ubiquitous in our everyday lives, HCI researchers and experts from behavioral sciences have been working in parallel and in collaborations to develop interventions that support and sustain behavior change\cite{hekler2013mind}. Other applications use theories from behavioral economics, like Nudge theory\cite{lee2011mining,gunaratne2015informing,caraban201923}, in implementing persuasive technologies\cite{fogg1998persuasive,fogg2002persuasive,fogg2009behavior,kim2012awareness,lee2015understanding,savage2016botivist} that use persuasion and social influence in order to change a person's behavior. However, Fogg clarifies that these techniques are different from those that coerce users into committing an action\cite{fogg2002persuasive}. 

Behavior theories and persuasive techniques have been used extensively in technologies that support health and fitness outcomes\cite{cowan2013apps,lister2014just} and other life domains\cite{michie2013behavior}. For example, they have been used to inform the design of whole applications that support healthy eating\cite{hsu2014persuasive,coughlin2015smartphone,okumus2016factors,lessel2016watercoaster,aydin2017save,reinhardt2019only}, regular exercise\cite{consolvo2006design,consolvo2008activity,consolvo2009goal,fritz2014persuasive,herrmanny2016using,de2016crowd} and good health maintenance\cite{purpura2011fit4life,stawarz2014don,orji2017towards}. They are also used in applications that help users reduce stress\cite{konrad2015finding,gimpel2015mystress}, stop smoking\cite{abroms2011iphone,heffner2015feature} and improve privacy decisions\cite{harbach2014using}. Aside from whole applications, behavior theories have also been used to design specific features of an application\cite{stawarz2014don,heffner2015feature}. 

Although these theories may seem easy to implement using a number of techniques, an application's design and implementation can only achieve a certain level of persuasion and extent of behavior change\cite{redstrom2006persuasive}. In some occasions, designers would opt for techniques that would maximize persuasion and social influence for quick attainment of behavior change outcomes. Despite well-meaning, this often translates to deceitful strategies like the \textit{dark patterns} in UX\cite{adar2013benevolent,gray2018dark}. This led to ethical considerations that designers must consider in their design of behavior change applications and persuasive technologies, especially for socio-technical systems. In the process of developing behavior change applications, designers make decisions that are guided by values which are either already operational or still latent to them. To help designers discover values they embed in their decisions, Chivukula et. al. introduced the method of Ethicography\cite{chivukula2019analyzing}. Through this method of value discovery, they found their participants inconsistently and indirectly referencing user-centered values. This resulted to designs that enhanced persuasion as opposed to user agency. This might explain the frequent use of deceit in HCI which can result to patterns of breakdown\cite{brynjarsdottir2012sustainably}. 

To avoid the risk of deceiving users into performing tasks, Self-Determination Theory provides a framework that allows us to inform our designs towards one's self-determined action and better quality motivation. Studies have looked at existing behavior change applications, classified the techniques they used and some mapped them to behavior change constructs\cite{lister2014just,stawarz2014don,cowan2013apps}. However, none of the classifications were mapped to SDT, which seems to suggest that although Self-Determination Theory is a prominent theory for human motivation, commercial applications have yet to inform their designs based on it. In HCI, SDT is most commonly used in games and play research, especially for serious games. Recently, Tyack and Mekler conducted a systematic review of HCI games research which aims to understand how SDT advanced the sub-field and how researchers engage with the theory\cite{tyack2020self}. They found that SDT-based game designs were mostly focused on needs satisfaction and intrinsic motivation. The other mini-theories, like the Causal Orientation Theory, were rarely engaged with.  

In this dissertation, I focus on using behavior change techniques that values user agency and that supports internalization towards higher quality of motivation. In the following chapter, I describe the formative study that deepened my understanding of the driving navigation task. This helped identify potential challenges to developing motivation for the selection of unselfish routes.