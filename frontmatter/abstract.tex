Advances in machine learning and recommender systems have promoted the ubiquity and positive utility of navigation applications in our daily navigation and commuting experiences. While they are commonly used when going to unknown destinations, navigation applications are also widely used in daily commutes to avoid congested roads in urban areas. And governments see this as a potential solution for their citizens to adopt sustainable routes and driving behaviors. This can be possible with a system that helps avoid traffic congestion in the road network by distributing drivers to different paths. But this "smart city" approach face the challenge of encouraging drivers who only rely on some or no route guidance. Current routing algorithms and voice guidance of navigation applications prioritize fastest routes, but it is unclear how these route guidance are adopted and what factors affect their adoption.

In this dissertation, I focus on the critical HCI question of how to encourage drivers to follow system optimal routes for their daily commutes. I explore novel and persuasive interaction techniques and information design that supports making informed decisions during navigation towards unselfish, purposeful, and effective driving behaviors. I begin with evaluating the features of existing commercial navigation systems and applications based on how they support self-efficacy, sense of agency and behavior change. I follow this with an observational study of drivers using navigation applications in their daily commutes. We found that while drivers choose a recommended route in urgent situations, many still preferred to follow familiar routes. They also made deviations from their original choice because of unfamiliar roads, lack of local context, perceived driving unsuitability, and inconsistencies with realized navigation experiences. From these initial works, I conducted two studies that evaluated prototypes that aim to help drivers make better informed navigation decisions. The first study explores the effects of adding explanations and other cues to route recommendations and route information when a driver chooses a route to follow before driving. During a trip, traffic conditions along a chosen route might change. The second study explores the use of two-party conversations between voice agents in giving alternative turns or routes. We found that while most drivers followed directions appropriate for the given scenarios, they were more likely to make inappropriate choices after hearing alternatives in conversations. On the other hand, two-party conversations allowed drivers to better reflect on their choices after trips. In a third study, I used Self-Determination Theory and insights on road familiarity from the second study to explore different combinations of navigational information and display them to the driver before starting a trip. An online within-subject study was conducted to evaluate the effectiveness of those additional navigational information in encouraging drivers to choose an unselfish route. In a final study, I integrated the display of pre-trip navigational information with the delivery of two-party conversations as voice guidance. Here, I aim to evaluate whether my combined techniques can encourage drivers to choose an unselfish route and continue following it until they reach their destination. I conclude with a discussion of how navigation solutions can influence the mobility patterns of drivers towards more livable and sustainable cities.